%%% --------------- DO NOT CHANGE THE FOLLOWING SECTION --------------------
\documentclass[12pt]{article}
\usepackage[margin=1in]{geometry}
\usepackage{amsmath}
\usepackage{amssymb}
\usepackage{amsthm}
\usepackage{graphicx}
\usepackage{hyperref}
\usepackage{tikz}
\usepackage{tikz-3dplot}
\usepackage{pgfplots}
\usetikzlibrary{arrows.meta,calc,3d}
\usepackage{fancyhdr}
\usepackage{multirow, multicol}
\usepackage{enumitem}
\usepackage{tabu}
\usepackage{hyperref}
\usepackage{dsfont}
\usepackage{mathrsfs}
\usepackage{stmaryrd}
\usepackage{mathtools}
\usepackage{biblatex}

\pgfplotsset{compat=1.17}
\tdplotsetmaincoords{65}{110}

\newtheorem{defn}{Definition}
\newtheorem{theorem}{Theorem}
\newtheorem{prop}{Proposition}
\newtheorem{lemma}{Lemma}
\newtheorem{corr}{Corollary}
\newtheorem{remark}{Remark}
\newtheorem{example}{Example}
\newtheorem{assmptn}{Assumption}
\theoremstyle{definition}
\newtheorem*{notation}{Notation}
\usepackage{comment}
\newenvironment{solution}{
  \textbf{Solution.}}{}
\renewcommand*{\thefootnote}{\fnsymbol{footnote}\fnsymbol{footnote}}

 \setlength{\headheight}{14.5pt}
 \addtolength{\topmargin}{-2.5pt}


%% ------------------ EXOTIC COMMANDS --------------------------------------

%----------------- Widebar -------------------------------------------------------------------------------------------------------------------------------------------------------------------------------------------------
%See https://tex.stackexchange.com/questions/16337/can-i-get-a-widebar-without-using-the-mathabx-package?noredirect=1&lq=1

\makeatletter
\let\save@mathaccent\mathaccent
\newcommand*\if@single[3]{%
  \setbox0\hbox{${\mathaccent"0362{#1}}^H$}%
  \setbox2\hbox{${\mathaccent"0362{\kern0pt#1}}^H$}%
  \ifdim\ht0=\ht2 #3\else #2\fi
  }
%The bar will be moved to the right by a half of \macc@kerna, which is computed by amsmath:
\newcommand*\rel@kern[1]{\kern#1\dimexpr\macc@kerna}
%If there's a superscript following the bar, then no negative kern may follow the bar;
%an additional {} makes sure that the superscript is high enough in this case:
\newcommand*\widebar[1]{\@ifnextchar^{{\wide@bar{#1}{0}}}{\wide@bar{#1}{1}}}
%Use a separate algorithm for single symbols:
\newcommand*\wide@bar[2]{\if@single{#1}{\wide@bar@{#1}{#2}{1}}{\wide@bar@{#1}{#2}{2}}}
\newcommand*\wide@bar@[3]{%
  \begingroup
  \def\mathaccent##1##2{%
%Enable nesting of accents:
    \let\mathaccent\save@mathaccent
%If there's more than a single symbol, use the first character instead (see below):
    \if#32 \let\macc@nucleus\first@char \fi
%Determine the italic correction:
    \setbox\z@\hbox{$\macc@style{\macc@nucleus}_{}$}%
    \setbox\tw@\hbox{$\macc@style{\macc@nucleus}{}_{}$}%
    \dimen@\wd\tw@
    \advance\dimen@-\wd\z@
%Now \dimen@ is the italic correction of the symbol.
    \divide\dimen@ 3
    \@tempdima\wd\tw@
    \advance\@tempdima-\scriptspace
%Now \@tempdima is the width of the symbol.
    \divide\@tempdima 10
    \advance\dimen@-\@tempdima
%Now \dimen@ = (italic correction / 3) - (Breite / 10)
    \ifdim\dimen@>\z@ \dimen@0pt\fi
%The bar will be shortened in the case \dimen@<0 !
    \rel@kern{0.6}\kern-\dimen@
    \if#31
      \overline{\rel@kern{-0.6}\kern\dimen@\macc@nucleus\rel@kern{0.4}\kern\dimen@}%
      \advance\dimen@0.4\dimexpr\macc@kerna
%Place the combined final kern (-\dimen@) if it is >0 or if a superscript follows:
      \let\final@kern#2%
      \ifdim\dimen@<\z@ \let\final@kern1\fi
      \if\final@kern1 \kern-\dimen@\fi
    \else
      \overline{\rel@kern{-0.6}\kern\dimen@#1}%
    \fi
  }%
  \macc@depth\@ne
  \let\math@bgroup\@empty \let\math@egroup\macc@set@skewchar
  \mathsurround\z@ \frozen@everymath{\mathgroup\macc@group\relax}%
  \macc@set@skewchar\relax
  \let\mathaccentV\macc@nested@a
%The following initialises \macc@kerna and calls \mathaccent:
  \if#31
    \macc@nested@a\relax111{#1}%
  \else
%If the argument consists of more than one symbol, and if the first token is
%a letter, use that letter for the computations:
    \def\gobble@till@marker##1\endmarker{}%
    \futurelet\first@char\gobble@till@marker#1\endmarker
    \ifcat\noexpand\first@char A\else
      \def\first@char{}%
    \fi
    \macc@nested@a\relax111{\first@char}%
  \fi
  \endgroup
}
\makeatother
\newcommand\test[1]{%
$#1{M}$ $#1{A}$ $#1{g}$ $#1{\beta}$ $#1{\mathcal A}^q$
$#1{AB}^\sigma$ $#1{H}^C$ $#1{\sin z}$ $#1{W}_n$}


%------------------------ Laplace Transform --------------------------------
%See https://tex.stackexchange.com/questions/239791/is-this-laplace-transform-symbol-available-in-latex

\newsavebox\foobox
\newlength{\foodim}
\newcommand{\slantbox}[2][0]{\mbox{%
        \sbox{\foobox}{#2}%
        \foodim=#1\wd\foobox
        \hskip \wd\foobox
        \hskip -0.5\foodim
        \pdfsave
        \pdfsetmatrix{1 0 #1 1}%
        \llap{\usebox{\foobox}}%
        \pdfrestore
        \hskip 0.5\foodim
}}
\def\Ell{\slantbox[-.45]{$\mathscr{L}$}}

%% -------------------------------------------------------------------------

%% ------------------------ FORMAT -----------------------------------------

\parindent 0pt
\parskip 10pt

\setcounter{section}{-1}
\pagestyle{fancy}
\hypersetup{
    colorlinks=true,
    linkcolor=blue,
    filecolor=magenta,      
    urlcolor=cyan,
    }
    
% Title
    \title{Stochastic Calculus}
    \date{}
    \author{Ansh S.}
    \addbibresource{bibliography.bib}

% \fancyhead[RO]{MATH-UA 377 Fall 2022}
\fancyhead[RO]{Stochastic Calculus}
\fancyhead[LO]{}
\newenvironment{problems}
{\begin{enumerate}[itemsep=30pt]}
  {\end{enumerate}}
\newenvironment{parts}
{\begin{enumerate}[topsep=10pt,itemsep=20pt,label*=\arabic*.]}
  {\end{enumerate}}


\begin{document}
\maketitle

%% -------------------------------------------------------------------------

%%% -------END OF SECTION THAT MUST BE LEFT UNCHANGED ----------------------


%%% ------------------- SHORTCUTS ------------------------------------------

\newcommand{\R}{\mathbb{R}}
\newcommand{\Z}{\mathbb{Z}}
\newcommand{\Q}{\mathbb{Q}}
\newcommand{\T}{\mathbb{T}}
\newcommand{\Rtilde}{\widetilde{\R}}
\newcommand{\Rvec}{\widehat{\R}}
\newcommand{\id}{\mathds{1}}
\newcommand{\N}{\mathbb{N}}
\newcommand{\C}{\mathbb{C}}
\newcommand{\B}{\mathcal{B}}
\newcommand{\power}{\mathcal{P}}
\newcommand{\M}{\mathcal{M}}
\newcommand{\schwartz}{\mathcal{S}}
\newcommand{\dynkin}{\mathcal{D}}
\newcommand{\A}{\mathcal{A}}
\newcommand{\F}{\mathcal{F}}
\newcommand{\cl}{\mathcal{C}}
\newcommand{\E}{\mathcal{E}}
\newcommand{\g}{\mathfrak{g}}
\newcommand{\no}{\textbf}
\newcommand{\problem}[1]{\textbf{Problem #1}}

\newcommand{\p}{\mathbb{P}}
\newcommand{\e}{\mathbb{E}}
\newcommand{\var}{\operatorname{Var}}
\newcommand{\cov}{\operatorname{Cov}}

\newcommand{\proj}{\operatorname{proj}}
\newcommand{\im}{\operatorname{im}}
\newcommand{\tr}{\operatorname{Tr}}
\newcommand{\ric}{\operatorname{Ric}}
\newcommand{\hess}{\operatorname{Hess}}

\let\oldemptyset\emptyset
\let\emptyset\varnothing

%\overset{(d)}{\longrightarrow}

%%% ------------------------------------------------------------------------

The goal of these notes is to study Stochastic Quantization, and to take the shortest path to getting there.

In the context of ODEs we've seen success in treating autonomous quasi-linear equations which take the form
\begin{align*}
  x^{(n)}(t)&=F(x^{(0)}(t),\cdots,x^{(n-1)}(t)).
\end{align*}
where $x:[0,\infty)\rightarrow\R^d$ can be thought of as the configuration of a physical system. In what follows, we will think of $x$ as the position of some particle.

Let's specialize to the following $1$-st order ODE:
\begin{align}
  x'(t)&=V(x(t))\label{eq:ode}
\end{align}
where $V:\R^d\rightarrow\R^d$ is a smooth vector field.
Since we are working in a flat space, there is no difference between the space in which position and velocity vectors live, however this changes in general in Riemannian geometry for instance and should be kept in mind.

The dynamics induced by \eqref{eq:ode} are deterministic. We can view the time evolution of the system as an iterative scheme where at each time $t$ the position $x(t)$ updates the velocity $x'(t)=V(x(t))$ at that time, which then in turn drives the energy further.
But what if we were trying to model phenomena where at each time $t$, there were random effects at play which biased the velocity $V(x(t))$ to some $V(x(t))+\xi(t)$, where $\xi$ is supposed to reflect the random kicks to the particle which disturb the velocity.
This is where we start our study of SDEs -- Stochastic Differential equations.

The above suggests looking (more generally) at
\begin{align}
  X'(t)&= V(X(t))+B(X(t))\xi(t),
\end{align}
which at this point only formally makes sense. Here, $\xi$ denotes {\it white noise} and $B:\R^d\rightarrow\R^d$ is the (current position-dependent) bias which controls how the noise disturbs the velocity.

\end{document}